\documentclass[11pt,letterpaper,twoside]{article}
\usepackage[hmarginratio=1:1,top=1in,bottom=.75in,left=1.5in,right=1.5in]{geometry}

\usepackage{etoolbox}
\usepackage[]{ccicons}
\usepackage{fontawesome5}
\usepackage{marginnote}
\usepackage{sidenotes}

\usepackage{needspace}
\usepackage[defaultlines=3,all]{nowidow}
\raggedbottom

\usepackage{ragged2e}
\setlength{\RaggedRightParindent}{0em}
\RaggedRight

\usepackage[singlespacing]{setspace}

\usepackage[none]{hyphenat}

\usepackage{parskip}
\setlength{\parindent}{0em}
\setlength{\parskip}{0.6em plus 0.2em minus 0.1em}
\setlength{\emergencystretch}{3em}  % prevent overfull lines

% ==========================
% = Language & Date Format =
% ==========================

\usepackage[english]{babel}
\usepackage{currfile}

% ==================
% = Colors & Fonts =
% ==================

\usepackage[table]{xcolor}
\definecolor{OffBlack}{HTML}{191919}
\definecolor{Maroon}{HTML}{800000}
\definecolor{TarHeelBlue}{HTML}{4b9cd3}
\definecolor{HopkinsBlue}{HTML}{002D72}
\definecolor{RacingGreen}{HTML}{004225}

\usepackage{fontspec}
\setmainfont{Athelas}[Mapping=tex-text]
\setsansfont{Adelle Sans}[Mapping=tex-text]
\setmonofont[Scale=0.90]{LFT Etica Mono}

% font style for margin notes
%\renewcommand{\marginfont}{\color{RacingGreen}\sffamily\scriptsize}

% smaller sizes for footnotes & captions
%\renewcommand{\footnotesize}{\scriptsize}
%\renewcommand{\captionsize}{\small\itshape}

% =======================
% = Headers and Footers = 
% =======================

\usepackage{fancyhdr}
\pagestyle{fancy}
\fancyfoot{}
\fancyfoot[C]{}
\fancyfoot[R]{\tiny\texttt{ Revised: \today}}
\fancyfoot[L]{}
\fancyhead[CE]{\textsc{Employment Law}}
\fancyhead[CO]{\textsc{Elon University School of Law}}
\fancyhead[LE]{\textsc{\thepage}}
\fancyhead[LO]{\textsc{Winter 2026}}
\fancyhead[RE]{\textsc{Prof. Fink}}
\fancyhead[RO]{\textsc{\thepage}}

% =========
% = Lists =
% =========

\usepackage{enumitem}
\setlistdepth{6}
\setlist[itemize]{leftmargin={1.5em}}
\providecommand{\tightlist}{%
\setlength{\itemsep}{0em}\setlength{\parskip}{0em}\setlength{\itemindent}{0em}}

% ============
% = Headings =
% ============

\usepackage[compact]{titlesec}
% \titlespacing*{<command>}{<left>}{<before-sep>}{<after-sep>}
\titlespacing*{\section}{0em}{0em}{.5em}
\titlespacing*{\subsection}{0em}{2em}{.5em}
\titlespacing*{\subsubsection}{0em}{1.5em}{.5em}
\titlespacing*{\paragraph}{0em}{1em}{0em}
\titlespacing*{\subparagraph}{0em}{.2em}{0em}

\setcounter{secnumdepth}{0}

% H1
\titleformat{\section}[block]
  {\huge\sffamily\color{Maroon}}
  {}
  {0em}
  {}

% H2
\titleformat{\subsection}[block]
  {\needspace{9\baselineskip}\LARGE\sffamily\color{Maroon}}
  {}
  {0em}
  {}
% H3
\titleformat{\subsubsection}[block]
  {\needspace{6\baselineskip}\Large\rmfamily\bfseries}
  {\thesubsection}
  {0em}
  {}
% H4
\titleformat{\paragraph}[block]
  {\footnotesize\sffamily\large\rmfamily\bfseries}
  {\thesubsubsection}
  {0em}
  {}
% H5
\titleformat{\subparagraph}[block]
  {\footnotesize\rmfamily}
  {}
  {0em}
  {}

% ==========
% = Quotes =
% ==========

\usepackage{csquotes}

\AtBeginEnvironment{quote}{\footnotesize\itshape} 

\renewenvironment{quote}{%
   \list{}{%
     \leftmargin2em   % this is the adjusting screw
     \rightmargin\leftmargin\parsep .1em }
   \item\relax
}
{\endlist}

% ==========
% = Tables =
% ==========

\usepackage{longtable,booktabs,array}
%\setlength\heavyrulewidth{0pt}
\usepackage{multirow}
\usepackage{calc} % for calculating minipage widths
% Correct order of tables after \paragraph or \subparagraph
\makeatletter
\patchcmd\longtable{\par}{\if@noskipsec\mbox{}\fi\par}{}{}
\makeatother
% Allow footnotes in longtable head/foot
\IfFileExists{footnotehyper.sty}{\usepackage{footnotehyper}}{\usepackage{footnote}}
\makesavenoteenv{longtable}
% set table font size & space between rows
\renewcommand{\arraystretch}{1.5}
\AtBeginEnvironment{longtable}{\footnotesize}

% =============
% = Footnotes = 
% =============

\usepackage[hang,flushmargin]{footmisc} 

%\renewcommand{\footnotesize}{\fontsize{9pt}{11pt}\selectfont} %use to change size of footnotes

% ==============
% = Hyperlinks =
% ==============

\usepackage[unicode=true]{hyperref}
\hypersetup{%
  colorlinks=true,
  allcolors=HopkinsBlue,
  breaklinks=true,
  pdfusetitle=true,
  pdfauthor={Eric M. Fink},
  pdftitle={Syllabus},
    pdfsubject={Employment Law},
      pdfproducer=LateX via pandoc,
  pdfcreator=LateX via pandoc
}

% ===============================
% ===== BEGIN DOCUMENT ==========
% ===============================

\begin{document}\thispagestyle{empty}

% Sets default font color without overriding hyperlink colors
\color{OffBlack}

% \maketitle

\section{Employment Law}

\begin{footnotesize}
\subparagraph{Elon University School of Law}
\subparagraph{Winter 2026}
\subparagraph{Room 107}
\subparagraph{Mondays \& Wednesdays, 10:30 am--12:15 pm}
\subparagraph{\url{emfink.net/EmploymentLaw}}
\vspace{1em}

\paragraph{Professor}
\subparagraph{Eric M. Fink} 
\subparagraph{efink@elon.edu}
\subparagraph{336.279.9334} 
\subparagraph{Office Hours: {\url{calendly.com/emfink/}}}

\vspace{1em}

\end{footnotesize}

\vspace{1em}

\needspace{1\baselineskip}

\subsection*{Description}\label{description}
\addcontentsline{toc}{subsection}{Description}

This course surveys federal and state laws governing employment, in both
the individual and organized labor settings. Topics to be covered
include establishing an employment relationship; recruitment \& hiring;
the scope and limits of supervisory control; confidentiality \&
competition; wages \& hours; employee health \& safety; and termination
of employment.

After completing the course, you should be able to recognize and
diagnose legal issues arising in the employment context, analyze those
issues under the applicable law, and help clients avoid legal problems
or pursue remedies when they arise. Simulation problems, including
in-class discussion and take-home assignments, provide an opportunity to
develop practical skills for representing clients in employment matters.

\subsection*{Course Materials}\label{course-materials}
\addcontentsline{toc}{subsection}{Course Materials}

\subsubsection{Required}\label{required}

Employment Law: An Open-Source Casebook (version 4.1, December 2025).
Available on the course website:
\href{http://www.emfink.net/EmploymentLaw}{emfink.net/EmploymentLaw}.

Rachel Arnow-Richman \& Nantiya Ruan, Developing Professional Skills:
Workplace Law (West Academic 2017) (``Workbook'').

\subsubsection{Suggested}\label{suggested}

Ann Hodges \& Rafael Gely, Principles of Employment Law (West Academic
2016). Digital version available at no cost through the
\href{https://subscription.westacademic.com/}{West Academic Online Study
Aids Collection}.

\subsection*{Policies}\label{policies}
\addcontentsline{toc}{subsection}{Policies}

\subsubsection{Grading}\label{grading}

Your final grade for this course will be based on two group assignments
(20\% each), one individual writing assignment (30\%), and class
participation (30\%). There will be no final exam.

The group assignments will be based on problems from the Developing
Professional Skills workbook. Each group will be responsible for two
problems during the term, as indicated in the Schedule and Assignments
section of the syllabus. I will provide further information about each
assignment before the due dates.

The individual writing assignment will be based on a problem that I will
provide in the last week of class. The due date for this assignment will
be March 21 (last day of Winter Term exam period).

\subsubsection{Attendance}\label{attendance}

Elon Law School has adopted the following attendance policy for all
courses:

\begin{quote}
The Law School administers a policy that a student maintain regular and
punctual class attendance in all courses in which the student is
registered, including externships, clinical courses, or simulation
courses. Faculty members will give students written notice of their
attendance policies before or during the first week of class. These
policies may include, but are not limited to: treating late arrivals,
early departures, and/or lack of preparation as absences imposing grade
or point reductions for absences, including assigning a failing grade or
involuntarily withdrawing a student from the class and any other
policies that a professor deems appropriate to create a rigorous and
professional classroom environment.

In case of illness or emergency, students may contact the Office of
Student and Professional Life, which will then notify the student's
instructors. A student may notify the faculty member directly of a
planned absence and should refer to individual faculty members regarding
any policy that may apply. In the case of prolonged illness or
incapacity, the student should contact the
\href{https://www.elon.edu/u/law/students/}{Office of Student Life}.
\end{quote}

You should let me know (in advance if feasible) if you are unable to
attend class, will arrive late, or must leave early. I do not require an
explanation of the reason, nor do I require a doctor's note or other
documentation.

\subsubsection{Disability
Accommodations}\label{disability-accommodations}

For disability accommodation requests, contact the
\href{https://www.elon.edu/u/academics/koenigsberger-learning-center/disabilities-resources/homepage/graduate-student-resources/}{Elon
Disability Services Office}.

\subsubsection{Honor Code}\label{honor-code}

The Law School
\href{https://www.elon.edu/u/law/students/honor-code/}{honor code}
applies to all activities related to your law school study, including
conduct during class and examinations.

\subsection*{Schedule \& Assignments}\label{schedule-assignments}
\addcontentsline{toc}{subsection}{Schedule \& Assignments}

(\emph{Note:} This class will not meet on Monday, January 20th)

\subsubsection{1. Foundations of Employment
Law}\label{foundations-of-employment-law}

\paragraph{January 5}\label{january-5}

Employment as a Socio-Legal Relationship, \emph{Casebook} Chap. 1, § 1

\paragraph{January 7}\label{january-7}

Employment as a Socio-Legal Relationship, \emph{Casebook} Chap. 1, § 1

\paragraph{January 12}\label{january-12}

Labor Organizing \& Collective Bargaining, \emph{Casebook} Chap. 2

\subsubsection{2. Establishing an Employment
Relationship}\label{establishing-an-employment-relationship}

\paragraph{January 14}\label{january-14}

Identifying Employees, \emph{Casebook} Chap. 3, § 1

\paragraph{January 21}\label{january-21}

Identifying Employers, \emph{Casebook} Chap. 3, § 2

\paragraph{January 26}\label{january-26}

Recruitment \& Hiring, \emph{Casebook} Chap. 4

\subsubsection{3.Scope \& Limits of Employer
Control}\label{scope-limits-of-employer-control}

\paragraph{January 28}\label{january-28}

Privacy, Autonomy, \& Dignity, \emph{Casebook} Chap. 5, § 1

\paragraph{February 2}\label{february-2}

Protected Concerted Activity, \emph{Casebook} Chap. 5, § 2

\paragraph{February 4}\label{february-4}

Control Outside Work, \emph{Casebook} Chap. 5, § 3

\subsubsection{4. Employee Duties to
Employers}\label{employee-duties-to-employers}

\paragraph{February 9}\label{february-9}

Confidentiality, \emph{Casebook} Chap. 6, § 1

\paragraph{February 11}\label{february-11}

Loyalty, \emph{Casebook} Chap. 6, § 2

\paragraph{February 16}\label{february-16}

Competition, \emph{Casebook} Chap. 6, § 2

\subsubsection{5. Wages \& Hours}\label{wages-hours}

\paragraph{February 18}\label{february-18}

Minimum Wage, Overtime, \& Wage Payments, \emph{Casebook} Chap. 7

\subsubsection{6. Employee Health \&
Safety}\label{employee-health-safety}

\paragraph{February 23}\label{february-23}

Medical Leave \& Disability Discrimination, \emph{Casebook} Chap. 8, §§
1 \& 2

\paragraph{February 25}\label{february-25}

OSHA \& Workers' Compensation, \emph{Casebook} Chap. 9, §§ 3 \& 4

\subsubsection{7. Terminating Employment}\label{terminating-employment}

\paragraph{March 2}\label{march-2}

Contractual Exceptions to the At-Will Presumption, \emph{Casebook} Chap.
10, § 1

\paragraph{March 4}\label{march-4}

Tort \& Statutory Claims for Wrongful Termination, \emph{Casebook} Chap.
10, §§ 2 \& 3

\paragraph{March 9}\label{march-9}

Unemployment Compensation, \emph{Casebook} Chap. 10, § 4

\paragraph{March 11}\label{march-11}

\subparagraph{Make-up Class (if needed)}\label{make-up-class-if-needed}

\end{document}
